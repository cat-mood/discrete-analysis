\section{Тест производительности}

Тест производительности представляет из себя следующее: 
КМП сравнивается с наивным алгоритмом на 3 тестах с паттерном длиной в 10 символов и текстом длиной в $10^5$, 
входные данные из себя представляют случайный набор букв "a" и "b".

\begin{alltt}
cat_mood@nuclear-box:~/programming/mai-da-labs/lab04/build$ ./lab04_benchmark < ../tests/e2e/test01.txt 
Naive: 6026 ms
KMP: 3682 ms
cat_mood@nuclear-box:~/programming/mai-da-labs/lab04/build$ ./lab04_benchmark < ../tests/e2e/test02.txt 
Naive: 4915 ms
KMP: 3953 ms
cat_mood@nuclear-box:~/programming/mai-da-labs/lab04/build$ ./lab04_benchmark < ../tests/e2e/test03.txt 
Naive: 5539 ms
KMP: 3542 ms
\end{alltt}

Как видно, КМП в среднем работает быстрее наивный алгоритм. 

Это связано с тем, что наивный алгоритм работает в среднем за $O(p \cdot (t - p))$, где $p$ -- длина паттерна,
а $t$ -- длина текста.

Сложность КМП в среднем равна $O(p + t)$, где $p$ -- длина паттерна,
а $t$ -- длина текста.
\pagebreak
