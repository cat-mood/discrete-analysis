\section{Тест производительности}

Тест производительности представляет из себя следующее: 
Patricia Trie сравнивается с \textit{std::map} на 3 тестах с разным количеством входных данных от $10^3$ до $10^5$, 
входные данные из себя представляют случайный набор команд добавления и поиска.

\begin{alltt}
[info] [2024-04-24 18:40:06] Running tests/e2e/01.t
std::map ms=23750
patricia ms=26714
[info] [2024-04-24 18:40:07] Running tests/e2e/02.t
std::map ms=23274
patricia ms=19875
[info] [2024-04-24 18:40:07] Running tests/e2e/03.t
std::map ms=7341
patricia ms=5843
\end{alltt}

Как видно, Patricia Trie в среднем работает за то же время, что и \textit{std::map}. 

Это связано с тем, что \textit{std::map} работает
на красно-чёрном дереве. Сложность поиска и вставки для него -- $O(\log n)$, где $n$ -- количество элементов в дереве. Так как ключом
является строка, то сложность поиска и вставки становится равной $O(k \log n)$, где $k$ -- длина ключа.

Сложность вставки и поиска в Patricia Trie равна $O(h)$, где $h$ -- высота дерева. Получение $i$-того бита в строке обходится в $O(1)$. 
В конце поиска происходит полное сравнение ключей, поэтому сложность поиска и вставки равна $O(\max (h, k))$, где $k$ -- длина строки (ключа).
\pagebreak
