\section{Тест производительности}


Тест производительности представляет из себя следующее: 
алгоритм поразрядной сортровки сравнивается с \textit{std::stable\_sort} на 7 тестах с разным количеством входных данных, 
входные данные из себя представляют случайный набор ключей и значений.

\begin{alltt}
[info] [2024-03-10 21:21:15] Running tests/01.t
Count of lines is 0
Radix sort time: 7us
STL stable sort time: 0us
[info] [2024-03-10 21:21:15] Running tests/02.t
Count of lines is 1
Radix sort time: 8us
STL stable sort time: 1us
[info] [2024-03-10 21:21:15] Running tests/03.t
Count of lines is 10
Radix sort time: 22us
STL stable sort time: 9us
[info] [2024-03-10 21:21:15] Running tests/04.t
Count of lines is 100
Radix sort time: 147us
STL stable sort time: 134us
[info] [2024-03-10 21:21:15] Running tests/05.t
Count of lines is 1000
Radix sort time: 1315us
STL stable sort time: 1840us
[info] [2024-03-10 21:21:15] Running tests/06.t
Count of lines is 10000
Radix sort time: 13636us
STL stable sort time: 22433us
[info] [2024-03-10 21:21:15] Running tests/07.t
Count of lines is 100000
Radix sort time: 140472us
STL stable sort time: 279561us
\end{alltt}

Как видно, алгоритм поразрядной сортировки выигрывает по времени у \textit{std::stable\_sort}, так как её асимптотическая сложность $O(n \log n)$, когда 
у поразрядной сортировки -- $O(k n)$, где $k$ -- количество разрядов. При выбранном варианте ключей (автомобильные номера) $k = 6$, поэтому 
сложность приближается к $O(n)$.

\pagebreak
