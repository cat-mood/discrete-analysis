\section{Описание}
Требуется написать реализацию алгоритма поиска подстроки в строке Кнута-Морриса-Пратта.

Как сказано в \cite{Gusfield}: Предположим, что при некотором выравнивании Р около Т наивный алгоритм обнаружил совпадение первых 
i символов из Р с их парами из T, а при следующем сравнении было несовпадение. 
В этом случае наивный алгоритм сдвинет Р на одно место и начнет сравнение заново с левого конца Р. 
Но часто можно сдвинуть образец дальше. Например, если Р = abcxabcde и при текущем расположении Р и Т несовпадение нашлось в позиции 8 строки Р, 
то есть возможность сдвинуть Р на четыре места без пропуска вхождений Р в Т. Отметим, что это можно увидеть, ничего не зная о тексте T и расположении Р 
относительно Т. Требуется только место несовпадения в Р. Алгоритм Кнута-Морриса-Пратта, основываясь на таком способе рассуждений, 
и делает сдвиг больше, чем наивный алгоритм.

\pagebreak

\section{Исходный код}
Функция \textit{ZFunction} реализует Z-алгоритм (Z[i] -- длина наибольшей подстроки, начинающейся в i и совпадающей с префиксом строки).

Функция \textit{SPFunction} реализует префикс-функцию (sp[i] -- длина наибольшего собственного суффикса, который совпадает с префиксом строки).

Функция \textit{KMPFunction} реализует алгоритм Кнута-Морисса-Пратта.

Для реализации алфавита в виде слов используется структура \textit{TWord}. Для этой струтуры перегружены операторы \enquote{==} и \enquote{!=}.
\begin{longtable}{|p{7.5cm}|p{7.5cm}|}
\hline
\rowcolor{lightgray}
\multicolumn{2}{|c|} {kmp.h}\\
\hline
std::vector<int> ZFunction(const TString\& s)&Z-алгоритм.\\
\hline
std::vector<int> SPFunction(const TString\& s)&Префикс-функция.\\
\hline
std::vector<int> KMPFunction(const TString\& pattern, const TString\& text)&КМП-алгоритм.\\
\hline
\end{longtable}

\begin{lstlisting}[language=C++]
struct TWord {
	char word[MAX_WORD_LEN];
	int lineId, wordId;

	TWord();
	void Clear();
	char& operator[](int index);
	bool operator==(const TWord& rhs) const;
	bool operator!=(const TWord& rhs) const;
};

using TString = std::vector<TWord>;
\end{lstlisting}
\pagebreak

\section{Консоль}
\begin{alltt}
cat_mood@nuclear-box:~/programming/mai-da-labs/lab04/build$ ./lab04_exe < test.txt 
2, 1
5, 1
12, 1
18, 1
23, 1
30, 2
35, 2
38, 1
44, 2
51, 1
57, 1
62, 9
62, 16
62, 22
62, 32
62, 39
62, 45
62, 55
\end{alltt}
\pagebreak
