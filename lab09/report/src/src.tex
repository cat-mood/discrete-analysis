\section{Описание}
Требуется реализовать алгоритм Форда-Фалкерсона. Точнее алгоритм Эдмондса-Карпа.
Это модификация алгоритма Форда-Фалкерсона.

В \cite{Kormen} приведён алгоритм Форда-Фалкерсона: \enquote{Поиск максимального потока методом
Форда-Фалкерсона проводится последовательно. Вначале поток нулевой (и величина его равна нулю).
На каждом шаге мы увеличиваем значение потока. Для этого мы находим "дополняющий путь", по
которому можно пропустить ещё немного вещества, и используем его для увеличения потока. Этот
шаг повторяется, пока есть дополняющие пути.}. 

Алгоритм Эдмондса-Карпа заключается в том, чтобы искать кратчайший \enquote{дополняющий путь} с помощью
обхода в ширину (BFS).

Сложность такого алгоритма -- $O(V \cdot E^2)$, где $V$ -- количество вершин в графе, $E$ -- количество рёбер.

\pagebreak

\section{Исходный код}
\begin{longtable}{|p{7.5cm}|p{7.5cm}|}
\hline
\rowcolor{lightgray}
\multicolumn{2}{|c|} {main.cpp}\\
\hline
std::vector<TVertex> FindPath(const TGraph\& graph, TVertex from, TVertex to)&Поиск пути с помощью BFS.\\
\hline
\end{longtable}

\begin{lstlisting}[language=C++]
	using TVertex = uint16_t;
	using TWeight = uint64_t;
	using TGraph = std::vector<std::unordered_map<TVertex, TWeight>>;
\end{lstlisting}
\pagebreak

\section{Консоль}
\begin{alltt}
	cat-mood@nuclear-box:~/programming/mai-da-labs/lab08/build$ ./lab08_exe 
	6
	1 5
	0 1
	1 2
	2 3
	3 4
	4 5
	5
	2
	1 5
	0 1 
\end{alltt}
\pagebreak
