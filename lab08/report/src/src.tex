\section{Описание}
Требуется решить задачу с помощью жадных алгоритмов.

В \cite{Kormen} сказано: \enquote{Такой алгоритм делает на каждом шаге локально оптимальный
выбор, -- в надежде, что итоговое решение также окажется оптимальным}. 

Я предлагаю отсортировать исходный массив отрезков по левой и правой границе. Затем
итерироваться по нему и на каждом шаге искать самый большой отрезок, который покрывает ещё
непокрытую на предыдущих этапах часть. Соответственно, на каждом шаге нужно обновлять
границы непокрытого исходного отрезка.

Сложность такого алгоритма -- $O(n \cdot log(n))$, где $n$ -- количество отрезков.
Такая сложность получается, потому что я использую quick sort.

\pagebreak

\section{Исходный код}
\begin{longtable}{|p{7.5cm}|p{7.5cm}|}
\hline
\rowcolor{lightgray}
\multicolumn{2}{|c|} {main.cpp}\\
\hline
bool IndexComparator(const Segment\& a, const Segment\& b)&Компаратор для сортировки по индексу.\\
\hline
bool SegmentComparator(const Segment\& a, const Segment\& b)&
Компаратор для сортировки по левой и правой границе.\\
\hline
\end{longtable}

\begin{lstlisting}[language=C++]
	struct Segment {
    int64_t start;
    int64_t end;
    size_t index;

    Segment() = default;

    Segment(int64_t start, int64_t end, size_t index) : start{start}, end{end}, index{index} {}

    int64_t ComputeCover(int64_t left, int64_t m) {
        if (end < 0 || start > m) {
            return 0;
        }
        return std::min<int64_t>(m, end) - std::max<int64_t>(left, start);
    }
};
\end{lstlisting}
\pagebreak

\section{Консоль}
\begin{alltt}
	cat-mood@nuclear-box:~/programming/mai-da-labs/lab08/build$ ./lab08_exe 
	6
	1 5
	0 1
	1 2
	2 3
	3 4
	4 5
	5
	2
	1 5
	0 1 
\end{alltt}
\pagebreak
